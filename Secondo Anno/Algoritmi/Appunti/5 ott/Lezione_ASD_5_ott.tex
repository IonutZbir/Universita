\documentclass{article}
\usepackage{algorithm}
\usepackage{algpseudocode}
\usepackage{graphicx}
\usepackage{amsmath} % For matrices
\usepackage{geometry}

\title{Algoritmi - Lezione 1}
\author{Ionut Zbirciog}
\date{6 October 2023}

\begin{document}

\maketitle

\section{Algoritmi per risolvere Fibonacci}

\subsection{fib1 - Formula chiusa}
Uso la formula chiusa: 
\[F_n = \frac{1}{\sqrt{5}} \cdot (\phi^n - \hat{\phi}^n)\]
\begin{algorithm}
\caption{fibonacci1}
\begin{algorithmic}[1]
\Function{fibonacci2}{intero $n$} $\rightarrow$ intero
    \State $phi = 1.618$
    \State $phiSegnato = -0.618$ 
    \State \Return $0.447 * phi^n - phiSegnato^n$
\EndFunction
\end{algorithmic}
\end{algorithm}


\subsection{fib2 - Ricorsione}
\begin{algorithm}
\caption{fibonacci2}
\begin{algorithmic}[1]
\Function{fibonacci2}{intero $n$} $\rightarrow$ intero
  \If {$n \leq 2$}
    \State \Return $1$
  \Else
    \State \Return \Call{fibonacci2}{$n-1$} + \Call{fibonacci2}{$n-2$}
  \EndIf
\EndFunction
\end{algorithmic}
\end{algorithm}
Costo: 
\[T(n) = 2 + T(n-1) + T(n-2)\]
\[T(1) = T(2) = 1\]

\subsection{Albero della Ricorsione}
\begin{figure}[h]
  \centering
  \includegraphics[width=0.8\textwidth]{AlberoRicorsione.png}
  \caption{Albero della ricorsione di fibonacci2.}
  \label{fig:image-label}
\end{figure}
\begin{itemize}
\item Utile per risolvere la relazione di ricorrenza.
\item Nodi corrispondenti alle chiamate ricorsive.
\item Figli di un nodo corrispondono alle chiamate.
\item Il primo nodo è la radice.
\item I nodi interni hanno etichetta $2$.
\item Le foglie hanno etichetta $1$.
\end{itemize}


\subsubsection{Calcolo di $T(n)$}
Per calcolare $T(n)$:
\begin{itemize}
\item Contiamo il numero di foglie.
\item Contiamo il numero di nodi interni.
\end{itemize}

\subsubsection{Lemma 1}
Il numero di foglie dell'albero della ricorsione di Fibonacci è pari a $F_n$.

\subsubsection{Lemma 2}
Il numero di nodi interni di un albero in cui ogni nodo interno ha due figli è pari al numero di foglie - 1.
In totale, otteniamo:
\[T(n) = F_n + 2(F_{n-1}) = 3F_n - 2 = F_n = \phi^n\]
È lento perché continua a ricalcolare ripetutamente la soluzione dello stesso sottoproblema (crescita esponenziale).

\subsection{fib3 - Memorizzazione in Array}
\begin{algorithm}
\caption{fibonacci3}
\begin{algorithmic}[1]
\Function{fibonacci3}{intero $n$} $\rightarrow$ intero
  \State Sia $Fib$ un array di $n$ interi
  \State $Fib[1] = 1; Fib[2] = 1$
  \For{$i = 3$ to $n$}
    \State $Fib[i] = Fib[i-1] + Fib[i-2]$
  \EndFor
  \State \Return $Fib[n]$
\EndFunction
\end{algorithmic}
\end{algorithm}
\[T(n) = n + n + 3 = 2n + 3 \]
Algoritmo più veloce rispetto a fibonacci2 di 38 milioni di volte con crescita lineare.
Unica pecca: l'utilizzo della memoria, fib3 occupa in memoria uno spazio proporzionale a $n$.

\subsection{fib4 - Memorizzazione degli ultimi 2 valori}
\begin{algorithm}
\caption{fibonacci4}
\begin{algorithmic}[1]
\Function{fibonacci4}{intero $n$} $\rightarrow$ intero
  \State $a = 1; b = 1$
  \For{$i = 3$ to $n$}
    \State $c = a + b$
    \State $a = b$
    \State $b = c$
  \EndFor
  \State \Return $c$
\EndFunction
\end{algorithmic}
\end{algorithm}
\[T(n) = 4n + 2 \]
E' più lento di fib3? Andiamo a vedere.

\subsection{Notazione Asintotica}
\begin{itemize}
\item Esprimere $T(n)$ in modo QUALITATIVO.
\item Perdere un po' in PRECISIONE ma guadagnare SEMPLICITÀ.
\item Di $T(n)$ vogliamo descrivere come cresce al crescere di $n$:
  \begin{itemize}
  \item Ignoro costanti moltiplicative.
  \item Ignoro termini di ordine inferiore.
  \end{itemize}
\end{itemize}

\subsection{Esempi di Notazione Asintotica}
\begin{itemize}
\item $T(n) = 5n + 9 = O(n)$
\item $T(n) = 6n^2 + 8n - 13 = O(n^2)$
\end{itemize}

\subsection{fib5 - Matrici}
\begin{itemize}
\item fib4 non è il miglior algoritmo possibile.
\item È possibile dimostrare per induzione la seguente proprietà di matrici:
\[
\begin{bmatrix}
  1 & 1 \\
  1 & 0
\end{bmatrix}^n
=
\begin{bmatrix}
  F_{n+1} & F_n \\
  F_n & F_{n-1}
\end{bmatrix}
\]
\item Useremo questa proprietà per progettare un algoritmo più efficiente.
\end{itemize}

\textbf{Lemma:}
\[
\begin{bmatrix}
  1 & 1 \\
  1 & 0
\end{bmatrix}^n
=
\begin{bmatrix}
  F_{n+1} & F_n \\
  F_n & F_{n-1}
\end{bmatrix}
\]

\begin{algorithm}
\caption{fibonacci5}
\begin{algorithmic}[1]
\Function{fibonacci5}{int $n$} $\rightarrow$ int
  \State $M = \begin{bmatrix} 1 & 1 \\ 1 & 0 \end{bmatrix}$
  \For{$i = 1$ to $n - 1$}
    \State $M = M \cdot \begin{bmatrix} 1 & 1 \\ 1 & 0 \end{bmatrix}$
  \EndFor
  \State \Return $M[0][0]$
\EndFunction
\end{algorithmic}
Il tempo di esecuzione è ancora $O(n)$ ma possiamo usare le proprietà delle potenze e ottenere un risultato migliore.
\end{algorithm}


\subsection{Calcolo di Potenze}
\begin{itemize}
\item Possiamo calcolare la $n$-esima potenza elevando al quadrato la $\left(\frac{n}{2}\right)$-esima potenza.
\item Se $n$ è dispari, eseguiamo una ulteriore moltiplicazione.

\item Esempio:
  \[3^2 = 9,\ 3^4 = (9)^2 = 81,\ 3^8 = (81)^2 = 6561\]
  Abbiamo eseguito solo $3$ prodotti invece di $7$.
\end{itemize}

\subsection{fib6}
\begin{algorithm}
\caption{fibonacci6}
\begin{algorithmic}[1]
\Function{fibonacci6}{int $n$} $\rightarrow$ int
  \State $A = \begin{bmatrix} 1 & 1 \\ 1 & 0 \end{bmatrix}$
  \State $M = \text{potenzaDiMatrice}(A, n-1)$
  \State \Return $M[0][0]$
\EndFunction

\Function{potenzaDiMatrice}{matrice $A$, int $k$} $\rightarrow$ matrice
  \If{$k = 0$}
    \State \Return $\begin{bmatrix} 1 & 0 \\ 0 & 1 \end{bmatrix}$
  \Else
    \State $M = \text{potenzaDiMatrice}(A, \frac{k}{2})$
    \State $M = M \cdot M$
  \EndIf
  \If{$k$ dispari}
      \State $M = M \cdot A$
  \EndIf
  \State \Return $M$
\EndFunction
\end{algorithmic}
\end{algorithm}

Costo: $T(n) = O(\log_2 n)$, quindi è esponenzialmente più veloce di fibonacci3.

\subsection{Quanta memoria usa un algoritmo?}
\begin{itemize}
\item Algoritmo non ricorsivo: dipende dalla memoria ausiliaria utilizzata (variabili, array, strutture dati).
\item Algoritmo ricorsivo: dipende dalla memoria ausiliaria utilizzata da ogni chiamata e dal numero di chiamate che sono contemporaneamente attive.
\end{itemize}

\textbf{Nota:} Una chiamata usa sempre almeno memoria costante.

\subsection{Analisi Memoria Ausiliaria Fibonacci2}
\begin{itemize}
\item Chiamate attive formano un cammino $P$ radice-nodo.
\item $P$ ha al più $n$ nodi.
\item Ogni nodo/chiamata usa memoria costante.
\end{itemize}

\textbf{Risultato:} Spazio $O(n)$.

\subsection{Analisi Memoria Ausiliaria Fibonacci6}
\textbf{Risultato:} Spazio $O(\log n)$.

\subsection{Riepilogo}
\begin{tabular}{|c|c|c|}
  \hline
    & Tempo di Esecuzione & Occupazione di Memoria \\
  \hline
  fibonacci2& $O(\phi^n)$ & $O(n)$ \\
  \hline
  fibonacci3& $O(n)$ & $O(n)$ \\
  \hline
  fibonacci4& $O(n)$ & $O(1)$ \\
  \hline
  fibonacci5& $O(n)$ & $O(1)$ \\
  \hline
  fibonacci6& $O(\log_2n)$ & $O(\log_2n)$ \\
  \hline
\end{tabular}


\end{document}
