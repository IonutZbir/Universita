\section{3SAT è NP-Completo}

\paragraph*{Dimostrazione: }

Partiamo con formalizzare la tripla $(I_{SAT}, S_{SAT}, \pi_{SAT})$ del problema $SAT$:
\begin{itemize}
    \item [] {
        $
            I_{SAT} = \{f: \{vero, falso\}^n \rightarrow \{vero, falso\}\ \text{tale che f è in forma congiuntiva normale (CNF)}\}    
        $
    }
    \item [] {
        $
            S_{SAT}(f) = \{(b_{1}, b_{2}, \dots, b_{n}) \in \{vero, falso\}^n\}    
        $
    }
    \item []{
        \begin{align*}
            \pi_{SAT}(f, S_{SAT}(f)) &= \exists (b_{1}, b_{2}, \dots, b_{n}) \in S_{SAT}(f): f(b_{1}, b_{2}, \dots, b_{n}) = \text{vero}, \\
            &\quad \text{ossia, sostituendo in } f \text{ ogni occorrenza della variabile } x_{i} \text{ con il valore } b_{i} \\
            &\quad \text{(ed ogni occorrenza di } \lnot x_{i} \text{ con } \lnot b_{i}) \text{ per ogni } i = 1, \dots, n, \\
            &\quad \text{la funzione } f \text{ assume il valore vero}.
        \end{align*}  
    }
\end{itemize}

Adesso, formalizziamo la tripla $(I_{3SAT}, S_{3SAT}, \pi_{3SAT})$ del problema $3SAT$:
\begin{itemize}
    \item [] {
        $
            I_{3SAT} = \{f: \{vero, falso\}^n \rightarrow \{vero, falso\}\ \text{tale che f è in forma 3-congiuntiva normale (3CNF)}\}    
        $
    }
    \item [] {
        $
            S_{3SAT}(f) = \{(b_{1}, b_{2}, \dots, b_{n}) \in \{vero, falso\}^n\}    
        $
    }
    \item []{
        \begin{align*}
            \pi_{3SAT}(f, S_{3SAT}(f)) &= \exists (b_{1}, b_{2}, \dots, b_{n}) \in S_{3SAT}(f): f(b_{1}, b_{2}, \dots, b_{n}) = \text{vero}, \\
            &\quad \text{ossia, sostituendo in } f \text{ ogni occorrenza della variabile } x_{i} \text{ con il valore } b_{i} \\
            &\quad \text{(ed ogni occorrenza di } \lnot x_{i} \text{ con } \lnot b_{i}) \text{ per ogni } i = 1, \dots, n, \\
            &\quad \text{la funzione } f \text{ assume il valore vero}.
        \end{align*}  
    }
\end{itemize}


Sia $f(x) \in I_{3SAT} = C_{1} \land C_{2} \land \dots \land C_{m},\ \forall i = 1, \dots, m,\ |C_{i}| = 3$.

Sia $g(y) \in I_{SAT} = D_{1} \land D_{2} \land \dots \land D_{v}$.
\newline

Per dimostrare che $3SAT$ è \textbf{NPC}, dobbiamo fare la riduzione $SAT \leq 3SAT$. Ovvero $g \in SAT \Leftrightarrow f \in 3SAT$.

$\forall i = 1, \dots, m$, vediamo costruire $C_{i}$ a partire da $D_{i}$:

Chiamiamo \textit{letterale}, una varibaile $l \in \{x_{i}, \lnot x_{i}\}$

\begin{enumerate}
    \item {
        $D_{i}$ contiene un solo letterale $l$. 
        \begin{align*}
            C_{i} = (l \lor z_{i1} \lor z_{i2}) \land (l \lor \lnot z_{i1} \lor z_{i2})
                  \land (l \lor z_{i1} \lor \lnot z_{i2}) \land (l \lor \lnot z_{i1} \lor \lnot z_{i2})
        \end{align*}
    }
    \item {
        $D_{i}$ contiene 2 letterali $l_{i1}\ \lor l_{i2}$. 
        \begin{align*}
            C_{i} = (l_{i1} \lor l_{i2} \lor z_{i1}) \land (\lnot z_{i1} \lor l_{i1} \lor l_{i2})
        \end{align*}
    }
    \item {
        $D_{i}$ contiene 3 letterali $l_{i1} \lor l_{i2} \lor l_{i3}$, allora $C_{i} = D_{i}$. 
    }
    \item {
        $D_{i}$ contiene 4 letterali $\underbrace{l_{i1} \lor l_{i2}} \lor \underbrace{l_{i3} \lor l_{i4}}$. In questo caso si raggruppano i primi 2
        e gli ultimi 2. 
        \begin{align*}
            C_{i} = (l_{i1} \lor l_{i2} \lor z_{i1}) \land (\lnot z_{i1} \lor l_{i3} \lor l_{i4})
        \end{align*}
    }
    \item {
        $D_{i}$ contiene $k \ge 4$ letterali $\underbrace{l_{i1} \lor l_{i2}} \lor \dots \lor \underbrace{l_{i(k-1)} \lor l_{ik}}$. In questo caso si raggruppano i primi 2
        e gli ultimi 2. 
        \begin{align*}
            C_{i} = (l_{i1} \lor l_{i2} \lor z_{i1}) \land (\lnot z_{i1} \lor l_{i3} \lor z_{i2})
            \land (\lnot z_{i2} \lor l_{i4} \lor z_{i3}) \land \dots \land (\lnot z_{i(k-3)} \lor l_{i(k-1)} \lor l_{ik})
        \end{align*}
    }
\end{enumerate}

Possiamo costruire $f$ a partire da $g$ in tempo $O(|f|^2)$ che è polinomiale. Sapendo già che 3SAT è \textbf{NP} e 
avendo trovato una riduzione da SAT a 3SAT, possiamo concludere che 3SAT è \textbf{NPC}.

