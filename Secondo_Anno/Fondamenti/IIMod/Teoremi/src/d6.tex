\section{Teoremi Dispensa 6}

\subsection{Teorema a pag. 3}

Sia $T$ una macchina di Turing deterministica, definita su un alfabeto $\Sigma \setminus \square$ e 
un insieme di stati $Q$, e sia $x\in \Sigma^{\star}$ tale che $T(x)$ termina, allora: 
$$dspace(T, x) \leq dtime(T, x) \leq dspace(T, x)|Q|(|\Sigma| + 1)^{dspace(T, x)}$$

\paragraph*{Dimostrazione:} 
\begin{enumerate}
    \item {
        $dspace(T, x) \leq dtime(T, x)$\\
        Banalmente, una computazione deterministica che termina in $k$ passi non può utilizzare più di $k$ celle del nastro.
    }
    \item {
        $dtime(T, x) \leq dspace(T, x)|Q|(|\Sigma| + 1)^{dspace(T, x)}$
        \begin{enumerate}
            \item {
                $dspace(T, x)|Q|(|\Sigma| + 1)^{dspace(T, x)}$: è il numero di stati globali possibili di $T$ nel caso in cui non più di $dspace(T, x)$ celle del nastro 
                vengano utilizzate dalla computazione $T(x)$.
            }
            \item {
                $(|\Sigma| + 1)^{dspace(T, x)}$: sono tutte le possibili parole di $dspace(T, x)$ simboli di $\Sigma \cup \{\square\}$, ossia tutte le possibili
                configurazioni delle $dspace(T, x)$ celle utilizzate.
            }
        \end{enumerate}
        Siano, dunque, $T$ una macchina deterministica e $x\in \Sigma^{\star}$ tali che $T(x)$ termina in $k$ passi utilizzando
        $dspace(T, x)$ celle del nastro. Poiché $T(x)$ termina in $k$ passi, essa è una successione di stati globali 
        $$SG_{0}(x), SG_{2}(x), \dots, SG_{k}(x)$$
        tali che $SG_{0}(x)$ è lo stato globale iniziale e per ogni $0 \leq i \leq k - 1$ esiste una transizione
        $SG_{i}(x) \rightarrow SG_{i + 1}(x)$, e  $SG_{k}(x)$ è lo stato globale finale.\\
        Sia $k(T, x) = dspace(T, x)|Q|(|\Sigma| + 1)^{dspace(T, x)}$. Se $T(x)$ durasse più di $k(T, x)$ passi (senza mai uscire
        dalle $dspace(T, x)$ celle), allora sarebbe una successione di stati globali contenente almeno due volte uno stesso stato globale, $SG_{h}$.
        
        \begin{figure}[h]
            \centering
            \begin{tikzpicture}[node distance=2cm, every node/.style={draw, circle}, >={Stealth[round]}]
                % Nodes
                \node (A) {$SG_{1}$};
                \node (B) [right of=A] {$SG_{2}$};
                \node (C) [right of=B] {$SG_{h}$};
                \node (D) [right of=C] {$SG_{h + 1}$};
                \node (E) [right of=D] {$SG_{h + 2}$};
                \node (F) [below of=E] {$SG_{k(T, x)}$};
                
                % Edges
                \draw[->] (A) -- (B);
                \draw[->] (B) -- (C);
                \draw[->] (C) -- (D);
                \draw[->] (D) -- (E);
                \draw[->] (E) -- (F);
                \draw[->] (F) -- (C);
            \end{tikzpicture}
        \end{figure}
        Ma $T$ è deterministica, allora, a partire da $SG_{h}$ è possibile eseguire un'unica quintupla (verso $SG_{h + 1}$) ed
        essa viene eseguita tutte le volte in cui $T(x)$ si trova in $SG_{h}$. Quindi, entrambe le volte, avviene una transizione
        verso lo stesso stao globale $SG_{h + 1}$, in questo modo $T(x)$ va in loop e non termina che è contro l'ipotesi che termina.
    }
\end{enumerate}

\newpage
\subsection{Teorema a pag. 4}

Sia $f: \mathbb{N} \rightarrow \mathbb{N}$ una funzione totale e calcolabile.
\begin{itemize}
    \item []{
        Se $L \subseteq \Sigma^{\star}$ è accettato da una macchina di Turing non deterministica $NT$ tale che,
        per ogni $x\in L$, $ntime(NT, x) \leq f(|x|)$ allora $L$ è decidibile.
    }
    \item []{
        Se $L \subseteq \Sigma^{\star}$ è accettato da una macchina di Turing non deterministica $NT$ tale che,
        per ogni $x\in L$, $nspace(NT, x) \leq f(|x|)$ allora $L$ è decidibile.
    }
\end{itemize}

\paragraph*{Dimostrazione:} Poiché $f$ è totale e calcolabile, esiste una macchina di Turing $T_{f}$ trasduttore tale che, per 
ogni $n \in \mathbb{N}$, $T_{f}(n)$ termina con il valore $f(n)$ scritto sul nastro di output. Assumiamo che l'input e l'output 
siano codificati in unario. Sia $L \subseteq \Sigma^{\star}$ un linguaggio accettato da una macchina di Turing $NT$ tale che 
$\forall x \in L,\ ntime(NT, x) \leq f(|x|)$. Deriviamo ora da $NT$ e $T_{f}$ una nuova macchina non deterministica $NT^{'}$ a tre nastri.
\begin{enumerate}
    \item [$N_{1}$:] viene scritto in unario l'input $x \in \Sigma^{\star}$.
    \item [$N_{2}$:] viene scritta la lunghezza di $x$ in unario.
    \item [$N_{3}$:] viene utitlizzato come clock, ovvero viene scritto $f(|x|)$.
\end{enumerate}
La computazione $NT^{'}$ consiste di 3 fasi:
\begin{enumerate}
    \item [FASE 1:]{
        $NT^{'}(x)$ scrive $|x|$ su $N_{2}$ in unario. Una volta letto $\square$ su $N_{1}$, le testine di $N_{1}$ e $N_{2}$
        vengono riposizionate sul carattere più a sinistra.
    }
    \item [FASE 2:]{
        Simula $T_{f}(|x|)$, usando $N_{2}$ come nastro di input e $N_{3}$ come nastro di output. Essa termina scrivendo il valore di 
        $f(|x|)$ su $N_{3}$ e riavvolge la testina.
    }
    \item [FASE 3:]{
        Simula i primi $f(|x|)$ passi della computazione, utilizzando $N_{1}$ come input e nastro di lavoro e $N_{3}$ come clock,
        ovvero come contatore del numero di istruzioni eseguite. Fino a quando viene letto 1 su $N_{3}$ viene eseguita un'istruzione di
        $NT(x)$ e la testina di $N_{3}$ viene spostata a destra. Se $NT(x)$ raggiunge $q_{A}$ o $q_{R}$, $NT^{'}(x)$ termina nel 
        medesimo stato. Se viene letto $\square$ su $N_{3}$, $NT^{'}(x)$ termina in $q_{R}$.
    }
\end{enumerate} 

Poiché $f$ è calcolabile e totale e poiché la simulazione della computazione $NT(x)$ nella terza fase viene forzatamente terminata,
se non ha terminato entro $f(|x|)$ passi, tutte le computazioni di $NT^{'}$ terminano.

\begin{itemize}
    \item {
        Se $x\in L$ allora poiché $NT$ accetta $x$ in $f(|x|)$ passi, nella terza fase termina in $q_{A}$ prima che venga
        letto $\square$.
    }
    \item {
        Se $x\notin L$ allora o $NT(x)$ termina in $q_{R}$ durante la terza fase e di conseguenza anche $NT^{'}(x)$ termina in $q_{R}$,
        oppure viene letto $\square$, ovvero $NT(x)$ non ha accettato $x$ in $f(|x|)$ passi e dunque $NT^{'}(x)$ rigetta.
    }
\end{itemize}
Questo dimostra che $NT^{'}$ decide $L$ e dunque che $L$ è decidibile.
\subsection{Teorema a pag. 10}

Per ogni funzione totale calcolabile $f: \mathbb{N} \rightarrow \mathbb{N}$ 
$$DTIME[f(n)] \subseteq NTIME[f(n)]\ \land\ DSPACE[f(n)] \subseteq NSPACE[f(n)]$$ 

\paragraph*{Dimostrazione:} Una macchina di Turing deterministica è una particolare macchina di Turing non deterministica 
avente grado di non determinismo pari a 1, inoltre ogni parola decisa in $k$ passi e anche accettata in $k$ passi 
(ogni parola decisa in $k$ celle e anche accettata in $k$ celle).

\newpage
\subsection{Teorema a pag. 10}

Per ogni funzione totale calcolabile $f: \mathbb{N} \rightarrow \mathbb{N}$ 

$$DTIME[f(n)] \subseteq DSPACE[f(n)]\ \land\ NTIME[f(n)] \subseteq NSPACE[f(n)]$$ 

\paragraph*{Dimostrazione:} Segue dal teorema:
$$dspace(T, x) \leq dtime(T, x)\dots$$
Sia $L \subseteq \Sigma^{\star}$ tale che $L \in DTIME[f(n)]$. Allora esiste $T$ che decide $L$ e tale che 
$\forall x\in \Sigma^{\star}$, $dtime(T, x) \in O(f(|x|))$. Poiché $dspace(T, x) \leq dtime(T, x) \in O(f(|x|))$,
allora $dspace(T, x) \in O(f(|x|))$, dunque $L \in DSPACE(f(|x|))$.\\ Analogamente per $NTIME[f(n)] \subseteq NSPACE[f(n)]$.

\subsection{Teorema a pag. 11}

Per ogni funzione totale calcolabile $f: \mathbb{N} \rightarrow \mathbb{N}$ 

$$DSPACE[f(n)] \subseteq DTIME[2^{O(f(n))}]\ \land\ NSPACE[f(n)] \subseteq NTIME[2^{O(f(n))}]$$

\paragraph*{Dimostrazione:} Segue dal teorema:
$$\dots dtime(T, x) \leq dspace(T, x)|Q|(|\Sigma| + 1)^{dspace(T, x)}$$
Sia $L \subseteq \Sigma^{\star}$ tale che $L \in DSPACE[f(n)]$. Allora, esiste una machina di Turing $T$ deterministica $T$
che decide $L$ e tale che, per ogni $x \in \Sigma^{\star}$, $dspace(T, x) \in O(f(|x|))$. Poiché:

Supponiamo che $\Sigma = \{0, 1\}$

\[
    \begin{aligned}[t]
    dtime(T, x) &\leq dspace(T, x)|Q|(|\Sigma| + 1)^{dspace(T, x)} \\
                 &\leq dspace(T, x)|Q|3^{dspace(T, x)}\\
                 &\leq 2^{\log(dspace(T,x))}\ |Q|\ 2^{\log(3)dspace(T, x)}\\
                 &\leq |Q|2^{\log(dspace(T,x))\ +\ \log(3)dspace(T, x)}\\
                 &\leq |Q|2^{(1 + \log(3))dspace(T, x)}
    \end{aligned}
\]

Allora $dtime(T, x)\in O(2^{O(f(|x|))})$ e dunque $L\in DTIME[2^{O(f(|x|))}]$

\subsection{Teorema a pag. 11}

Per ogni funzione totale calcolabile $f: \mathbb{N} \rightarrow \mathbb{N}$ 
$$DTIME[f(n)] = coDTIME[f(n)]\ \land\ DSPACE[f(n)] = coDSPACE[f(n)]$$

\paragraph*{Dimostrazione:} $\forall L \in DTIME[f(n)]$, esiste $T$ che decide $L$ e $dtime(T, x) \in O(f(|x|))$. 
Da $T$ deriviamo $T^{c}$ con input $x \in \Sigma^{\star}$ e $Q_{f} = \{q_{A}^{c}, q_{R}^{c}\}$ che decide $L^{c}$ nel seguente modo:

\begin{enumerate}
    \item [FASE 1:] Simula $T(x)$
    \item [FASE 2:]{
        \begin{itemize}
            \item Se $T(x) = q_{A}$, allora $T^{c}(x) = q_{R}$
            \item Se $T(x) = q_{R}$, allora $T^{c}(x) = q_{A}$
        \end{itemize}
    }
\end{enumerate}
Dunque $L^{c} \in DTIME[f(n)]$.\\
Analogamente possiamo dimostrare che un qualsiasi linguaggio $L \in coDTIME[f(n)]$. Di conseguenza che $DTIME[f(n)] = coDTIME[f(n)]$.
Dimostrazione analoga per $DSPACE[f(n)] = coDSPACE[f(n)]$.
\newpage
\subsection{Teorema a pag. 14}

Sia $f: \mathbb{N} \rightarrow \mathbb{N}$ una funzione time-constructible (space-constructible.).
\begin{itemize}
    \item []{
        Allora, per ogni $L \in NTIME[f(n)]$, si ha che $L$ è decidibile in tempo non deterministico in $O(f(n))$.
    }
    \item []{
        Allora, per ogni $L \in NSPACE[f(n)]$, si ha che $L$ è decidibile in spazio non deterministico in $O(f(n))$.
    }
\end{itemize}

\paragraph*{Dimostrazione:} Sia $f: \mathbb{N} \rightarrow \mathbb{N}$ una funzione time-constructible. Allora esiste
una macchina di Turing di tipo trasduttore $T_{f}$ che, avendo scritto sul nastro di input/lavoro $n \in \mathbb{N}$ in 
unario, in $O(f(n))$ passi scrive sul nastro di output il valore di $f(n)$ in unario. Sia $L \in NTIME[f(n)]$. Allora 
esiste una $NT$ che accetta $L$ e tale che, per ogni $x \in L$, $ntime(NT, x) \in O(f(|x|))$. Definiamo ora da $T_{f}$
e $NT$, la macchina $NT^{'}$ con input $x \in L$ che opera nel seguente modo:

\begin{itemize}
    \item [FASE 1:] Scrive su $N_{2}$, $|x|$ in unario. 
    \item [FASE 2:] Simula $T_{f}(x)$ e scrive il risulato in unari su $N_{3}$.
    \item [FASE 3:] {
        Finché legge 1 su $N_{3}$ esegue una istruzione di $NT(x)$ su $N_{1}$. 
        \begin{itemize}
            \item Se termina in $q_{A}$, allora $NT^{'}(x) = q_{A}$.
            \item Altrimenti sposta la testina a destra su $N_{3}$
        \end{itemize}
    }
    \item [FASE 4:] Se legge $\square$ su $N_{3}$ allora rigetta.
\end{itemize}

\begin{itemize}
    \item La fase 1 termina in $O(|x|)$ passi.
    \item la fase 2 termina in $O(f(|x|))$ passi, in quanto $f$ è time-constructible.
    \item La fase 3 termina in $O(f(|x|))$ passi, in quanto $\forall x \in \Sigma^{\star}$, $ntime(NT^{'}, x) \in O(f(|x|))$.
\end{itemize}

Dunque, $NT^{'}(x)$ decide $L,\ \forall x \in \Sigma^{\star}$, e $ntime(NT^{'}, x) \in O(f(|x|))$. 

\subsection{Teorema a pag. 14}

Per ogni funzione $f: \mathbb{N} \rightarrow \mathbb{N}$ time-constructible, $$NTIME[f(n)] \subseteq DTIME[2^{O(f(n)}]$$

\paragraph*{Dimostrazione:} 

Sia $L \in \Sigma^{\star}$ tale che $L \in NTIME[f(n)]$, allora esistono una macchina di Turing $NT$ che accetta $L$ e una 
costante $h$ tali che $\forall x \in L,\ ntime(NT, x) \leq hf(|x|)$. Poiché $f$ è time-constructible, esiste una macchina 
di Turing deterministica $T_{f}$ con inpun la rappresentazione in unario di $n \in \mathbb{N}$, che calcola il valore di 
$f(n)$ in unario in tempo $O(f(n))$. Indichiamo con $k$ il grado di non determinismo di $NT$ e definiamo $T$ deterministica
che simuli $NT$ con input $x \in \Sigma^{\star}$ che opera nel seguente modo:

\begin{itemize}
    \item [FASE 1:] Scrive su $N_{2}$, $|x|$ in unario.
    \item [FASE 2:] {
        Simula $T_{f}(|x|)$ e scrive su $N_{3}$ l'output $hf(|x|)$ in unario.
    }
    \item [FASE 3:] {
        Per ogni computazione deterministica $\alpha(x)$ in $NT(x)$:
        \begin{itemize}
            \item Finché legge 1 su $N_{3}$ esegue un'istruzione lungo $\alpha(x)$.
            \item Se $\alpha(x) = q_{A}$ allora $T(x) = q_{A}$.
            \item Altrimenti sposta la testina a destra su $N_{3}$.
            \item Se legge $\square$ si sposta sul primo 1 a sinistra e passa alla prossima computazione deterministica $\alpha(x)$ se esiste.
        \end{itemize}
    }
    \item [FASE 4:] Rigetta
\end{itemize}

\begin{itemize}
    \item La fase 1 termina in $O(|x|)$ passi.
    \item La fase 2 termina in $O(hf(|x|)) = O(f(|x|))$ passi.
    \item La fase 3 termina in $O(f(|x|)k^{hf(x)})$ poiché esegue ogni computazione deterministica di $NT(x)$
\end{itemize}

Dunque, $$dtime(T, x) \in O(f(|x|)k^{hf(x)}) \subseteq{O(2^{O(f(|x|))})}$$

\subsection{Teorema a pag. 20}

Siano $C$ e $C^{'}$ due classi di complessità tali che $C^{'} \subseteq C$. Se $C^{'}$ è chiusa rispetto ad una $\pi$-riduzione
allora, per ogni linguaggio $L$ che sia $C$-completo rispetto a tale $\pi$-riduzione, $L \in C^{'}$ se e solo se $C = C^{'}$.

\paragraph*{Dimostrazione:}

\begin{itemize}
    \item [$\Leftarrow$)] Se $C = C^{'}$ allora $L \in C^{'}$.
    \item [($\Rightarrow$]{
        Sia $L \in C^{'}$. Poiché $L$ è $C$-completo rispetto alla $\pi$-riducibilità, allora per ogni $L_{0} \in C$, 
        $L_{0} \leq_{\pi} L$. Poiché $C^{'}$ è chiusa rispetto alla $\pi$-riduzione, ovvero per ogni altro linguaggio 
        $L^{'}$, $L^{'} \leq_{\pi} L$, allora $L^{'} \in C^{'}$, dunque per ogni linguaggio $L_{0} \in C$, $L_{0} \leq_{\pi} L$,
        allora $L_{0} \in C^{'}$. Quindi $C = C^{'}$.
    }
\end{itemize}

\subsection{Teorema a pag. 21}

La classe \textbf{P} è chiusa rispetto alla riducibilità polinomiale.

\paragraph*{Dimostrazione:}

Sia $L \in \textbf{P}$, allora esistono una macchina di Turing $T$ deterministica e $k \in \mathbb{N}$ tale che $T$ decide $L$ e 
per ogni $x \in \Sigma^{\star}$, $dtime(T, x) \in O(|x|^k)$.

Sia $L^{'}:\ L^{'} \leq_{p} L$, allora esiste una funzione $f: \Sigma^{\star}_{1} \rightarrow \Sigma^{\star}_{2}$ in $\textbf{FP}$
che riduce $L^{'}$ a $L$, con $T_{f}$ trasduttore tale che per ogni $x \in \Sigma^{\star}_{1}$, $T_{f}(x) \in L \Leftrightarrow x \in L^{'}
\land dtime(T_{f}, x) \in O(|x|^c)$.

Da $T$ e $T_{f}$ definiamo $T^{'}$ con input $x$ che opera nel seguente modo:

\begin{itemize}
    \item [FASE 1:] Simula $T_{f}(x)$ e scrive l'output $y$ su $N_{2}$.
    \item [FASE 2:] {
        Simula $T(y)$:
        \begin{itemize}
            \item Se termina in $q_{A}$ allora $T^{'}$ accetta.
            \item Se termina in $q_{R}$ allora $T^{'}$ rigetta.
        \end{itemize}
    }
\end{itemize}

Dato che $f$ è una riduzione da $L^{'}$ a $L$, quindi $f(x) \in L \Leftrightarrow x \in L^{'}$, quindi $T^{'}$ termina per 
ogni input $x \in \Sigma^{\star}$ e accetta $\Leftrightarrow$ $T(f(x))$ accetta, ossia $\Leftrightarrow\ f(x) \in L$.

Resta da mostrare che $T^{'}(x)$ opera in tempo polinomiale in $|x|$. La simulazione di $T_{f}(x)$ richiede 
$dtime(T_{f}, x) \leq |x|^c$ e la simulazione di $T(f(x))$ richiede $dtime(T, f(x)) \leq |f(x)|^k$.

\[
    \begin{aligned}[t]
        dtime(T^{'}, x) &\leq |x|^c + f(|x|)^k\footnote{$f(|x|)^k = |y|^k$ che per il teorema: $dspace(T, x) \leq dtime(T, x)$, 
        $|y|^k \leq (|x|^c)^k = |x|^{ck}$} \\
                &\leq |x|^c + |x|^{ck}\\
                &\Rightarrow dtime(T^{'}, x) \in O(|x|^{ck})
    \end{aligned}
\]

Poiché $c$ e $k$ sono costanti, allora risulta che \textbf{P} è chiusa rispetto la riducibilità polinomiale dato che $L^{'} \in \textbf{P}$.

\subsection{Teorema a pag. 21}

Le classi \textbf{NP, PSPACE, EXPTIME, NEXPTIME}, sono chiuse rispetto alla riducibilità polinomiale.

\paragraph*{Dimostrazione:} 

\begin{itemize}
    \item [\textbf{NP}]
    \item [\textbf{PSPACE}]
    \item [\textbf{EXPTIME}]
    \item [\textbf{NEXPTIME}]
\end{itemize}

\subsection{Corollario a pag. 21}

Se \textbf{P $\neq$ NP} allora, per ogni linguaggio \textbf{NP}-completo $L$, $L \notin \textbf{P}$.

\paragraph*{Dimostrazione:} 

Supponiamo che $L$ sia un linguaggio \textbf{NP}-completo e che $L \in \textbf{P}$. Poiché $L$ è \textbf{NP}-completo, per ogni
linguaggio $L^{'} \in \textbf{NP},\ L^{'} \leq L$, ma se $L \in \textbf{P}$, poiché \textbf{P} è chiusa rispetto alla 
riduzione $\leq$, questo implica che, per ogni $L^{'} \in \textbf{NP},\ L^{'} \in \textbf{P}$. Ossia \textbf{P = NP}, 
contraddicendo l'ipotesi.

\subsection{Teorema a pag. 23}

Se \textbf{NP $\neq$ coNP}, allora \textbf{P $\neq$ NP}.

\paragraph*{Dimostrazione ($A\rightarrow B \Leftrightarrow \lnot B \rightarrow \lnot A$): }

Se \textbf{P = NP}, allora \textbf{NP = coP} poiché \textbf{P = coP}. Ma se \textbf{P = NP} $\land$ \textbf{P = coP}, allora 
\textbf{coP = coNP}. Dunque: $$\textbf{\underline{NP} = P = coP = \underline{coNP}}$$ 

\subsection{Teorema a pag. 23}

La classe \textbf{coNP} è chiusa rispetto alla riducibilità polinomiale.

\paragraph*{Dimostrazione: }

Poiché \textbf{NP} è chiusa rispetto alla riducibilità polinomiale
\begin{itemize}
    \item [] Per ogni $L_{2} \in \textbf{NP}$ e per ogni $L_{1}$, se $L_{1} \leq L_{2}$, allora $L_{1} \in \textbf{NP}$.
    \item [$\Rightarrow$] Per ogni $L_{2}^{c} \in \textbf{coNP}$ e per ogni $L_{1}^{c}$, se $L_{1}^{c} \leq L_{2}^{c}$, allora $L_{1}^{c} \in \textbf{coNP}$
\end{itemize}

\subsection{Teorema a pag. 23}

Un linguaggio L è \textbf{NP}-completo se e soltanto se $L^{c}$ è \textbf{coNP}-completo. 

\paragraph*{Dimostrazione: }

\begin{itemize}
    \item [($\Rightarrow$]{
        Sia $L$ un linguaggio \textbf{NP}-completo, allora per definizione di completezza
        \begin{enumerate}
            \item $L \in \textbf{NP}$ allora $L^{c} \in \textbf{coNP}$.
            \item $\forall L_{0} \in \textbf{NP},\ L_{0} \leq L$.
        \end{enumerate}
        Sia $L_{1}$ un qualsiasi linguaggio in \textbf{coNP}, allora $L^{c}_{1} \in \textbf{NP}$. 
        Poiché $L$ è \textbf{NP}-completo, $L_{1}^{c} \leq L$.
        \[
            \begin{aligned}[t]
                &\Rightarrow Esiste\ f \in \textbf{FP}: \forall x \in \Sigma^{\star} [x \in L^{c}_{1} \Leftrightarrow f(x) \in L]\\
                &\Rightarrow Esiste\ f \in \textbf{FP}: \forall x \in \Sigma^{\star} [x \notin L^{c}_{1} \Leftrightarrow f(x) \notin L]\\
                &\Rightarrow Esiste\ f \in \textbf{FP}: \forall x \in \Sigma^{\star} [x \in L_{1} \Leftrightarrow f(x) \in L^{c}]\\
            \end{aligned}
        \]
        In conclusione, $\forall L_{1} \in \textbf{coNP}:\ L_{1} \leq L^{c}$, e quindi $L^{c}$ è \textbf{coNPC}.
    }
    \item [$\Leftarrow$)]{
        Sia $L^{c}$ un linguaggio \textbf{coNP}-completo, allora per definizione di completezza
        \begin{enumerate}
            \item $L^{c} \in \textbf{coNP}$ allora $L \in \textbf{NP}$.
            \item $\forall L_{0}^{c} \in \textbf{coNP},\ L_{0}^{c} \leq L^{c}$.
        \end{enumerate}
        Sia $L_{1}$ un qualsiasi linguaggio in \textbf{NP}, allora $L^{c}_{1} \in \textbf{coNP}$.
        Poiché $L^{c}$ è \textbf{coNP}-completo, $L_{1}^{c} \leq L^{c}$.
        \[
            \begin{aligned}[t]
                &\Rightarrow Esiste\ f \in \textbf{FP}: \forall x \in \Sigma^{\star} [x \in L^{c}_{1} \Leftrightarrow f(x) \in L^{c}]\\
                &\Rightarrow Esiste\ f \in \textbf{FP}: \forall x \in \Sigma^{\star} [x \notin L^{c}_{1} \Leftrightarrow f(x) \notin L^{c}]\\
                &\Rightarrow Esiste\ f \in \textbf{FP}: \forall x \in \Sigma^{\star} [x \in L_{1} \Leftrightarrow f(x) \in L]\\
            \end{aligned}
        \]
        In conclusione, $\forall L_{1} \in \textbf{NP}:\ L_{1} \leq L$, e quindi $L$ è \textbf{NPC}.
    }
\end{itemize}

\newpage
\subsection{Teorema a pag. 24}

Se esiste un linguaggio $L$ \textbf{NP}-completo tale che $L \in \textbf{coNP}$, allora \textbf{NP = coNP}.

\paragraph*{Dimostrazione: }

$L$ è \textbf{NP}-completo, allora per definizione di completezza
\begin{enumerate}
    \item $L \in \textbf{NP}$.
    \item $\forall L_{1} \in \textbf{NP},\ L_{1} \leq L$.
\end{enumerate}

\begin{itemize}
    \item [$\subseteq$]{
        Se $L \in \textbf{coNP}$ allora $\forall L_{1} \in \textbf{NP},\ L_{1} \leq L$, ma \textbf{coNP} è chiusa ripsetto 
        alla riducibilità polinomiale ovvero [$L_{2} \in \textbf{coNP},\ L_{1} \leq L_{2},\ \Rightarrow\ L_{1} \in \textbf{coNP}$],
        allora, per ogni $L_{1} \in \textbf{NP}$ si ha che $L_{1} \leq L$, e $L \in \textbf{coNP}$. Dunque, per la chisura di \textbf{coNP},
        $L_{1} \in \textbf{coNP}$, quindi \textbf{NP $\subseteq$ coNP}. 
    }
    \item [$\supseteq$]{
        Poiché $L \in \textbf{coNP}$, allora $L^{c} \in \textbf{NP}$, ma poiché $L$ è \textbf{NP}-completo, allora $L^{c}$ è
        \textbf{coNP}-completo, quindi $\forall L^{'} \in \textbf{coNP},\ L^{'} \leq L^{c}$. Ma \textbf{NP} è chiusa rispetto alla 
        riducibilità polinomiale, ovvero [$L_{2} \in \textbf{NP},\ L_{1} \leq L_{2},\ \Rightarrow\ L_{1} \in \textbf{NP}$],
        allora, per ogni $L^{'} \in \textbf{coNP},\ L^{'} \leq L^{c}$ e $L^{c} \in \textbf{NP}$. Dunque, per la chiusura di \textbf{NP}, 
        $L^{'} \in \textbf{NP}$, quindi \textbf{coNP $\subseteq$ NP}.   
    }
\end{itemize}

In conclusione, se $L$ è \textbf{NP}-completo $\land$ $L \in \textbf{coNP}$, allora \textbf{NP = coNP}.