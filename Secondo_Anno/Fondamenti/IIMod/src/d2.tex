\section{Teoremi Dispensa 2}

\subsection{Teorema a pag. 5}
Per ogni macchina di Turing non deterministica $NT$ esiste una macchina di Turing detreministica $T$
tale che, per ogni possibile input $x$ di $NT$ , l'esito della computazione $NT(x)$ coincide con l'esito della computazione
di $T(x)$.

\paragraph*{Dimostrazione:}
Eseguiamo una simulazione della macchina non deterministica $NT$ mediante una macchina deterministica $T$. La simulazione consiste 
in una visita in ampiezza\footnote{Perché non in profondità? Non possiamo fare una visità in profondità perché non sappiamo la lunghezza di ciascuna computazione, in quanto potrebbero
anche non finire.} dell'albero delle computazioni di $NT$ basata sulla tecnica \textit{coda di rondine con ripetizioni}.
Partiamo dallo stato globale $SG(T, x, 0)$ e simuliamo tutte le computazione di lunghezza 1. Se tutte le computazioni terminano in 
$q_{R}$ allora $T$ rigetta, se almeno una computazione termina in $q_{A}$ allora $T$ accetta, altrimenti ricominciamo da capo 
eseguendo tutte le computazioni di lunghezza 2 e così via.