\section{Altre Dimostrazioni}

\subsection{\textbf{PSPACE} $\subseteq$ \textbf{EXPTIME}}

Dopo aver definito le classi $\textbf{PSPACE}\ e\ \textbf{EXPTIME}$ dimostrare che $\textbf{PSPACE} \subseteq \textbf{EXPTIME}$.

\paragraph*{Definizione:} 
\[
    \textbf{PSPACE} = \bigcup_{k\in \mathbb{N}} DSPACE[n^k]
\]
 
\[
    \textbf{EXPTIME} = \bigcup_{k\in \mathbb{N}} DTIME[2^{f(n, k)}]
\]

\paragraph*{Dimostrazione:} È una diretta conseguenza del seguente teorema: $\forall\ f$, totale e calcolabile

$$DSPACE[f(n)] \subseteq DTIME[2^{O(f(n))}]$$ 

A sua volta questo teorema è una diretta conseguenza del teorema

$$\dots \leq dtime(T, x) \leq dspace(T, x)|Q|(|\Sigma| + 1)^{dspace(T, x)}$$

Sia $L \subseteq \{0, 1\}^*$ tale che $L \in DSPACE[f(n)]$, allora $\exists T, k$ che decide $L$ e $dspace(T, x) \leq |x|^k$.
\[
    \begin{aligned}[t]
    dtime(T, x) &\leq dspace(T, x)|Q|(|\Sigma| + 1)^{dspace(T, x)} \\
                 &\leq dspace(T, x)|Q|3^{dspace(T, x)}\\
                 &\leq 2^{\log(dspace(T,x))}\ |Q|\ 2^{\log(3)dspace(T, x)}\\
                 &\leq |Q|2^{\log(dspace(T,x))\ +\ \log(3)dspace(T, x)}\\
                 &\leq |Q|2^{(1 + \log(3))dspace(T, x)}
    \end{aligned}
\]

In conclusione $dtime(T, x) \in O(2^{O(f(|x|))}) \Rightarrow L \in DTIME[2^{O(f(|x|))}]$

In questo modo abbiamo dimostrato che $\textbf{PSPACE} \subseteq \textbf{EXPTIME}$.

\subsection{\textbf{P} $\subseteq$ \textbf{PSPACE}}

Dopo aver definito le classi $\textbf{P}\ e\ \textbf{PSPACE}$ dimostrare che $\textbf{P} \subseteq \textbf{PSPACE}$.

\paragraph*{Definizione:} 

\[
   \textbf{P} = \bigcup_{k\in \mathbb{N}} DTIME[n^k]
\]

\[
    \textbf{PSPACE} = \bigcup_{k\in \mathbb{N}} DSPACE[n^k]
\]

\paragraph*{Dimostrazione:} È una diretta conseguenza del seguente teorema: $\forall\ f$, totale e calcolabile

$$DTIME[f(n)] \subseteq DSPACE[f(n)]$$ 

Che a sua volta è una diretta conseguenza del seguente teorema:
$$dspace(T, x) \leq dtime(T, x)\leq \dots$$

Sia $L \in \{0, 1\}^*,\ L \in DTIME[f(n)]:\ \exists T, k: \forall x \in \{0, 1\}^*\ T$ decide $L$ e $dtime(T, x) \leq f(n)$.

Poiché $dspace(T, x) \leq dtime(T, x) \land dtime(T, x) \in O(f(n)) \Rightarrow dspace(T,x) \in O(f(n))$.

Dunque $L \in DSPACE[f(n)]$.

\newpage

\subsection{\textbf{NP} $\subseteq$ \textbf{EXPTIME}}

Dopo aver definito le classi $\textbf{NP}\ e\ \textbf{EXPTIME}$ dimostrare che $\textbf{NP} \subseteq \textbf{EXPTIME}$.

\paragraph*{Definizione:} 

\[
   \textbf{NP} = \bigcup_{k\in \mathbb{N}} NTIME[n^k]
\]

\[
    \textbf{EXPTIME} = \bigcup_{k\in \mathbb{N}} DTIME[2^{f(n, k)}]
\]

\paragraph*{Dimostrazione:} È una diretta conseguenza del seguente teorema: $\forall\ f$, time-constructible

$$NTIME[f(n)] \subseteq DTIME[2^{O(f(n))}]$$

\textbf{Oss.} È una diretta conseguenza in quanto $n^k$ è una funzione time-constructible.

Sia $L \in \{0, 1\}^*$ tale che $L \in NTIME[f(n)]$, allora esistono una macchina di Turing $NT$ che accetta $L$ e una 
costante $h$ tali che $\forall x \in L,\ ntime(NT, x) \leq hf(|x|)$. Poiché $f$ è time-constructible, esiste una macchina 
di Turing deterministica $T_{f}$ con inpun la rappresentazione in unario di $n \in \mathbb{N}$, che calcola il valore di 
$f(n)$ in unario in tempo $O(f(n))$. Indichiamo con $k$ il grado di non determinismo di $NT$ e definiamo $T$ deterministica
che simuli $NT$ con input $x \in \Sigma^{\star}$ che opera nel seguente modo:

\begin{itemize}
    \item [FASE 1:] Scrive su $N_{2}$, $|x|$ in unario.
    \item [FASE 2:] {
        Simula $T_{f}(|x|)$ e scrive su $N_{3}$ l'output $hf(|x|)$ in unario.
    }
    \item [FASE 3:] {
        Per ogni computazione deterministica $\alpha(x)$ in $NT(x)$:
        \begin{itemize}
            \item Finché legge 1 su $N_{3}$ esegue un'istruzione lungo $\alpha(x)$.
            \item Se $\alpha(x) = q_{A}$ allora $T(x) = q_{A}$.
            \item Altrimenti sposta la testina a destra su $N_{3}$.
            \item Se legge $\square$ si sposta sul primo 1 a sinistra e passa alla prossima computazione deterministica $\alpha(x)$ se esiste.
        \end{itemize}
    }
    \item [FASE 4:] Rigetta
\end{itemize}

\begin{itemize}
    \item La fase 1 termina in $O(|x|)$ passi.
    \item La fase 2 termina in $O(hf(|x|)) = O(f(|x|))$ passi.
    \item La fase 3 termina in $O(f(|x|)k^{hf(x)})$ poiché esegue ogni computazione deterministica di $NT(x)$
\end{itemize}

Dunque, $$dtime(T, x) \in O(f(|x|)k^{hf(x)}) \subseteq{O(2^{O(f(|x|))})} \Rightarrow L \in DTIME[O(2^{O(f(n))})]$$

In conclusione $\textbf{NP} \subseteq \textbf{EXPTIME}$

\subsection{\textbf{P} = \textbf{coP} $\land$ \textbf{PSPACE} = \textbf{coPSPACE}}

Dopo aver definito le classi $\textbf{P, PSPACE, coP, coPSPACE}$ dimostrare che 

\textbf{P} = \textbf{coP} $\land$ \textbf{PSPACE} = \textbf{coPSPACE}.

\paragraph*{Definizione:} 

\[
   \textbf{P} = \bigcup_{k\in \mathbb{N}} DTIME[n^k]
\]

\[
   \textbf{coP} = \bigcup_{k\in \mathbb{N}} coDTIME[n^k]
\]

\[
    \textbf{PSPACE} = \bigcup_{k\in \mathbb{N}} DSPACE[n^k]
\]

\[
    \textbf{coPSPACE} = \bigcup_{k\in \mathbb{N}} coDSPACE[n^k]
\]

\paragraph*{Dimostrazione:} È una diretta conseguenza del seguente teorema: $\forall\ f$, totale e calcolabile
$$DTIME[f(n)] = coDTIME[f(n)]\ \land\ DSPACE[f(n)] = coDSPACE[f(n)]$$

$\forall L \in DTIME[f(n)]$, esiste $T$ che decide $L$ e $dtime(T, x) \in O(f(|x|))$. 
Da $T$ deriviamo $T^{c}$ con input $x \in \Sigma^{\star}$ e $Q_{f} = \{q_{A}^{c}, q_{R}^{c}\}$ che decide $L^{c}$ nel seguente modo:

\begin{enumerate}
    \item [FASE 1:] Simula $T(x)$
    \item [FASE 2:]{
        \begin{itemize}
            \item Se $T(x) = q_{A}$, allora $T^{c}(x) = q_{R}$
            \item Se $T(x) = q_{R}$, allora $T^{c}(x) = q_{A}$
        \end{itemize}
    }
\end{enumerate}
Dunque $L^{c} \in DTIME[f(n)]$.\\
Analogamente possiamo dimostrare che un qualsiasi linguaggio $L \in coDTIME[f(n)]$. Di conseguenza che $DTIME[f(n)] = coDTIME[f(n)]$.
Dimostrazione analoga per $DSPACE[f(n)] = coDSPACE[f(n)]$.